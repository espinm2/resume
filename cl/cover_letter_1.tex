%%%%%%%%%%%%%%%%%%%%%%%%%%%%%%%%%%%%%%%%%
% Plain Cover Letter
% LaTeX Template
% Version 1.0 (28/5/13)
%
% This template has been downloaded from:
% http://www.LaTeXTemplates.com
%
% Original author:
% Rensselaer Polytechnic Institute 
% http://www.rpi.edu/dept/arc/training/latex/resumes/
%
% License:
% CC BY-NC-SA 3.0 (http://creativecommons.org/licenses/by-nc-sa/3.0/)
%
%%%%%%%%%%%%%%%%%%%%%%%%%%%%%%%%%%%%%%%%%

%----------------------------------------------------------------------------------------
%	PACKAGES AND OTHER DOCUMENT CONFIGURATIONS
%----------------------------------------------------------------------------------------

\documentclass[11pt]{letter} % Default font size of the document, change to 10pt to fit more text
\usepackage{graphicx}

\usepackage{newcent} % Default font is the New Century Schoolbook PostScript font 
%\usepackage{helvet} % Uncomment this (while commenting the above line) to use the Helvetica font

% Margins
\topmargin=-1in % Moves the top of the document 1 inch above the default
\textheight=8.5in % Total height of the text on the page before text goes on to the next page, this can be increased in a longer letter
\oddsidemargin=-10pt % Position of the left margin, can be negative or positive if you want more or less room
\textwidth=6.5in % Total width of the text, increase this if the left margin was decreased and vice-versa

\let\raggedleft\raggedright % Pushes the date (at the top) to the left, comment this line to have the date on the right

\begin{document}

%----------------------------------------------------------------------------------------
%	ADDRESSEE SECTION
%----------------------------------------------------------------------------------------

\begin{letter}{
Recruitment Officer\\
Datto\\
Norwalk, CT 06851} 

%----------------------------------------------------------------------------------------
%	YOUR NAME & ADDRESS SECTION
%----------------------------------------------------------------------------------------

\begin{center}
\large\bf Max Espinoza \\ % Your name
%\vspace{20pt} \hrule height 1pt % If you would like a horizontal line separating the name from the address, uncomment the line to the left of this text
2317 15th Street \\ Troy, NY 12180 \\ (203) 646-6533 % Your address and phone number
\end{center} 


\signature{Max Espinoza} % Your name for the signature at the bottom

%----------------------------------------------------------------------------------------
%	LETTER CONTENT SECTION
%----------------------------------------------------------------------------------------

\opening{Dear Recruitment Officer,} 

% Where I learned about this place
I attended  Rensselaer Polytechnic Institute's Career Fair and met with an employee representing Datto. I was motivated by Datto's use of technology to back up client's data and allow virtual instances of saved machines to run remotely. I'm excited about your company and I’ve attached my resume, which contains more detail of my past projects and experience.

% Mini Purpose
Despite coming from a research heavy background, I have worked on web based applications before. 
I've implemented a web front end as part of my research project to gather 3D models of architectural spaces. 
In addition because the computational complexity of simulations we provide users, I implemented the back end of our site to interface with our research code. Allowing users to get models generated from efficient c++ implementations. I used WebGL to display this information in 3D. Moreover I used  PostgreSQL to save research relevant data gathered from users. 
Overall I think the skill set I have acquired from this project, which you can view through my web page, would be compatible with your summer internship program.

%Closing statement
You can reach me at (203) 464-6533 or through email at espinm2@rpi.edu. Thank you for taking time to consider my credentials for this summer internship position.


Sincerely yours,

\includegraphics[width=4cm]{../signature.png}

Max Espinoza

\encl{Resume} % List your enclosed documents here, comment this out to get rid of the "encl:"

%----------------------------------------------------------------------------------------

\end{letter}

\end{document}
